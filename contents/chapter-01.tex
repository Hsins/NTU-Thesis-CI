% !TeX root = ../main.tex

\chapter{緒論}

\section{前言}

臣亮言:先帝創業未半,而中道崩殂。今天下三分,益州疲弊,此誠危急存亡之秋也。然侍衛之臣,不懈於內;忠志之士,忘身於外者,蓋追先帝之殊遇,欲報之於陛下也。誠宜開張聖聽,以光先帝遺德,恢弘志士之氣;不宜妄自菲薄,引喻失義,以塞忠諫之路也。宮中府中,俱為一體,陟罰臧否,不宜異同。若有作姦犯科,及為忠善者,宜付有司,論其刑賞,以昭陛下平明之治,不宜篇私,使內外異法也。\par

侍中、侍郎郭攸之、費褘、董允等,此皆良實,志慮忠純,是以先帝簡拔以遺陛下。愚以為宮中之事,事無大小,悉以咨之,然後施行,必能裨補闕漏,有所廣益。將軍向寵,性行淑均,曉暢軍事,試用於昔日,先帝稱之曰「能」,是以眾議舉寵為督。愚以為營中之事,悉以咨之,必能使行陣和睦,優劣得所。親賢臣,遠小人,此先漢所以興隆也;親小人,遠賢臣,此後漢所以傾頹也。先帝在時,每與臣論此事,未嘗不歎息痛恨於桓、靈也。侍中、尚書、長史;參軍,此悉貞良死節之臣也,願陛下親之信之,則漢室之隆,可計日而待也。

臣本布衣,躬耕於南陽,苟全性命於亂世,不求聞達於諸侯。先帝不以臣卑鄙,猥自枉屈,三顧臣於草廬之中,諮臣以當世之事,由是感激,遂許先帝以驅馳。後值傾覆,受任於敗軍之際,奉命於危難之間,爾來二十有一年矣!先帝知臣謹慎,故臨崩寄臣以大事也。受命以來,夙夜憂勤,恐託付不效,以傷先帝之明。故五月渡瀘,深入不毛。今南方已定,兵甲已足,當獎率三軍,北定中原,庶竭駑鈍,攘除奸凶,興復漢室,還於舊都;此臣所以報先帝而忠陛下之職分也。至於斟酌損益,進盡忠言,則攸之、褘、允之任也。

願陛下託臣以討賊興復之效;不效,則治臣之罪,以告先帝之靈。若無興德之言,則戮允等,以彰其慢。陛下亦宜自課,以諮諏善道,察納雅言,深追先帝遺詔,臣不勝受恩感激。

今當遠離,臨表涕泣,不知所云。

\section{研究動機與目的}

對酒當歌,人生幾何!譬如朝露,去日苦多。
慨當以慷,憂思難忘。何以解憂?唯有杜康。
青青子衿,悠悠我心。但為君故,沉吟至今。
呦呦鹿鳴,食野之苹。我有嘉賓,鼓瑟吹笙。
明明如月,何時可掇?憂從中來,不可斷絕。
越陌度阡,枉用相存。契闊談宴,心念舊恩。
月明星稀,烏鵲南飛。繞樹三匝,何枝可依?
山不厭高,海不厭深。周公吐哺,天下歸心。

\section{論文架構}

自余爲僇人,居是州,恆惴慄。其隙也,則施施而行,漫漫而遊。日與其徒上高山,入深林,窮回溪;幽泉怪石,無遠不到。到則披草而坐,傾壺而醉,醉則更相枕以臥,臥而夢。意有所極,夢亦同趣。覺而起,起而歸。以爲凡是州之山有異態者,皆我有也,而未始知西山之怪特。

今年九月二十八日,因坐法華西亭,望西山,始指異之。遂命僕人過湘江,緣染溪,斫榛莽,焚茅茷,窮山之高而止。攀援而登,箕踞而遨,則凡數州之土壤,皆在衽席之下

其高下之勢,岈然窪然,若垤若穴,尺寸千里,攢蹙累積,莫得遁隱;縈青繚白,外與天際,四望如一。然後知是山之特出,不與培塿爲類。悠悠乎與灝氣俱,而莫得其涯;洋洋乎與造物者遊,而不知其所窮。

引觴滿酌,頹然就醉,不知日之入,蒼然暮色,自遠而至,至無所見,而猶不欲歸。心凝形釋,與萬化冥合。然後知吾向之未始遊,遊於是乎始,故爲之文以志。是歲,元和四年也。